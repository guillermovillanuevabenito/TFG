% !TEX root = ../my-thesis.tex
%
\chapter{Introduction}
\label{sec:intro}

\cleanchapterquote{You can’t do better design with a computer, but you can speed up your work enormously.}{Wim Crouwel}{(Graphic designer and typographer)}

\Blindtext[2][2]

\section{Postcards: My Address}
\label{sec:intro:address}

\textbf{Ricardo Langner} \\
Alfred-Schrapel-Str. 7 \\
01307 Dresden \\
Germany


\section{Motivation and Problem Statement}
\label{sec:intro:motivation}

\Blindtext[3][1] \cite{Jurgens:2000,Jurgens:1995,Miede:2011,Kohm:2011,Apple:keynote:2010,Apple:numbers:2010,Apple:pages:2010}
\cite{Prinz}

\section{Results}
\label{sec:intro:results}

\Blindtext[1][2]

\subsection{Some References}
\label{sec:intro:results:refs}

\cite{WEB:GNU:GPL:2010,WEB:Miede:2011}
\Blindtext[1][1]

\subsubsection{Methodology}
\label{sec:intro:results:refs:method}

\Blindtext[1][2]

\paragraph{Strategy 1}
\Blindtext[1][1]

\begin{lstlisting}[language=Python, caption={This simple helloworld.py file prints Hello World.}\label{lst:pyhelloworld}]
#!/usr/bin/env python
print "Hello World"
\end{lstlisting}

\paragraph{Strategy 2}
\Blindtext[1][1]

\begin{lstlisting}[language=Python, caption={This is a bubble sort function.}\label{lst:pybubblesort}]
#!/usr/bin/env python
def bubble_sort(list):
    for num in range(len(list)-1,0,-1):
        for i in range(num):
            if list[i]>list[i+1]:
                tmp = list[i]
                list[i] = list[i+1]
                list[i+1] = tmp

alist = [34,67,2,4,65,16,17,95,20,31]
bubble_sort(list)
print(list)
\end{lstlisting}

\section{Thesis Structure}
\label{sec:intro:structure}

\textbf{Chapter \ref{sec:related}} \\[0.2em]
\blindtext

\textbf{Chapter \ref{sec:system}} \\[0.2em]
\blindtext

\textbf{Chapter \ref{sec:concepts}} \\[0.2em]
\blindtext

\textbf{Chapter \ref{sec:concepts}} \\[0.2em]
\blindtext

\textbf{Chapter \ref{sec:conclusion}} \\[0.2em]
\blindtext

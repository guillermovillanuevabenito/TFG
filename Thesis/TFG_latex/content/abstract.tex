% !TEX root = ../my-thesis.tex
%
\pdfbookmark[0]{Abstract}{Abstract}
\addchap*{Abstract}
\label{sec:abstract}

Neural oscillatory patterns can be characterized by a number of attributes such as frequency, amplitude, duty cycle, characteristic transition times between silent and active phases, and number of spikes per burst. The value of these attributes are determined by the interplay of the participating currents and, for the appropriate currents, can be captured the maximal synaptic conductances. Experimental and theoretical work has shown that multiple combinations of parameters can generate patterns with the same attributes \cite{Prinz,Rot,Oly,Olypher2010}. This endows neurons and networks with flexibility to adapt to changing environments and is substrate for homeostatic regulation \cite{Olypher2010}. At the same time, it presents modelers with the phenomenon of unidentifiability in parameter estimation. Attribute Level sets (LSs) in parameter space are curves (surfaces or hypersurfaces) joining parameter values for which a given attribute is constant. Whether and under what circumstances the attribute LSs for individual neurons are conserved in the networks in which they are embedded and what additional network level sets emerge is not well understood.

In this work we describe a canonical (C-) model for oscillations LSs for single cells exhibiting a wide range of realistic neuronal oscillatory patterns. The model can be considered as an idealization of the familiar, conductance-based two-dimensional models. Under certain conditions, the LSs for individual C-cells are preserved in the network of C-cells. Moreover, new LSs emerge in these networks. We characterize them in terms of the single cell LSs and the connectivity parameters for both homogeneous and heterogeneous networks where individual cells are identical or not, respectively, within the considered LS.

\textbf{Keywords:} level sets, oscillations, neuronal networks.

\textbf{MSC2020:} 92B20


\newpage
\vspace*{5mm}
{\usekomafont{chapter}Resumen}
\label{sec:abstract-diff}
\vspace*{12mm}

Los patrones oscilatorios neuronales pueden estar caracterizados por un cierto número de atributos como la frequencia, la amplitud, el ciclo de trabajo, los tiempos característicos de transición entre fases silenciosas y activas, y el número de potenciales de acción por ráfaga. El valor de estos atributos está determinado por la interacción de las corrientes participantes y, para las corrientes apropiadas, se puede expresar en términos de las conductancias sinápticas máximas. Trabajo experimental y teórico ha mostrado que múltiples combinaciones de parámetros pueden generar patrones con los mismos atributos \cite{Prinz,Rot,Oly,Olypher2010}. Esto dota a las neuronas y redes con la flexibilidad para adaptarse a ambientes cambiantes y es fuente para la regulación homeostática \cite{Olypher2010}. Al mismo tiempo, los modeladores se encuentran con el fenómeno de inidentifiabilidad en la estimación de parámetros. Los conjuntos de nivel de los atributos (LSs) en el espacio de parámetros son curvas (superficies o hipersuperficies) que consisten en un conjuntos de valores de parámetros que mantienen constante un atributo. Si y bajo qué circunstancias los LSs de las neuronas individuales son conservados en las redes de las que forman parte y qué nuevos LSs emergen en la red no está bien entendido.

En este trabajo describimos un modelo canónico (C-) para LSs de oscilaciones de una célula individual, que exhibe un amplio rango de patrones oscilatorios realistas. El modelo puede ser considerado como una idealización de los familiares modelos de dos dimensiones basados en conductancias. Bajo ciertas condiciones, los LSs de una célula individual son preservados en redes compuestas por C-células. Además, nuevos LSs emergen en estas redes. Los caracterizamos en términos de los LSs de la célula individual y los parámetos de conectividad tanto en redes homogeneas como en heterogeneas donde los células son identicas o no, respectivamente, dentro del LS considerado.

\textbf{Palabras clave:} conjuntos de nivel, oscilaciones, redes neuronales.

\textbf{MSC2020:} 92B20

\newpage
\vspace*{5mm}
{\usekomafont{chapter}Resum}
\label{sec:abstract-diff}
\vspace*{12mm}

Els patrons oscil·latoris neuronals poden estar caracteritzats per un cert nombre d'atributs com la freqüència, l'amplitud, el cicle de treball, els temps característics de transició entre fases silencioses i actives, i el nombre de potencials d'acció per ràfega. El valor d'aquests atributs està determinat per la interacció dels corrents participants i, per als corrents apropiades, es pot expressar en termes de les conductàncies sinàptiques màximes. Treball experimental i teòric ha mostrat que múltiples combinacions de paràmetres poden generar patrons amb els mateixos atributs \cite{Prinz,Rot,Oly,Olypher2010}. Això dota les neurones i xarxes amb la flexibilitat per adaptar-se a ambients canviants i és font per a la regulació homeostàtica \cite{Olypher2010}. A el mateix temps, els modeladors es troben amb el fenomen de inidentifiabilitat en l'estimació de paràmetres. Els conjunts de nivell dels atributs (LSs) en l'espai de paràmetres són corbes (superfícies o hipersuperficies) que consisteixen en un conjunts de valors de paràmetres que mantenen constant un atribut. Si i sota quines circumstàncies els LSs de les neurones individuals són conservats en les xarxes de les que formen part i quins nous LSs emergeixen a la xarxa no està ben entès.

En aquest treball descrivim un model canònic (C-) per LSs d'oscil·lacions d'una cèl·lula individual, que exhibeix un ampli rang de patrons oscil·latoris realistes. El model pot ser considerat com una idealització dels familiars models de dues dimensions basats en conductàncies. Sota certes condicions, els LSs d'una cèl·lula individual són preservats en xarxes compostes per C-cèl·lules. A més, nous LSs emergeixen en aquestes xarxes. Els caracteritzem en termes dels LSs de la cèl·lula individual i els paràmetres de connectivitat tant en xarxes homogènies com en heterogènies on els cèl·lules són idèntiques o no, respectivament, dins l'LS considerat.

\textbf{Paraules clau:} conjunts de nivell, oscil·lacions, xarxes neuronals.

\textbf{MSC2020:} 92B20

% !TEX root = ../my-thesis.tex
%
\pdfbookmark[0]{Abstract}{Abstract}
\addchap*{Resumen}
\label{sec:abstract2}

Los patrones neuronales oscillatorios pueden ser caracterizados por un cierto número de atributos como la frequencia, la amplitud, el ciclo de trabajo, los tiempos característicos de transición entre fases silenciosas y activas, y el número de potenciales de acción por ráfaga. El valor de estos atributos está determinado por la interacción de las corrientes participantes y, para las corrientes apropiadas, se pueden expresar en términos de las conductancias sinápticas máximas. Trabajo experimental y teórico ha mostrado que múltiples combinaciones de parámetros pueden generar patrones con los mismos atributos. \cite{Prinz,Rot,Oly,Olypher2010}. Esto dota a las neuronas y redes con la flexibilidad para adaptarse a ambientes cambiantes y es fuente para la regulación homeostática \cite{Olypher2010}. Al mismo tiempo, los modeladores se encuentran con el fenómeno de inidentifiabilidad en la estimación de parámetros. Los conjuntos de nivel de los atributos (LSs) en el espacio de parámetros son curvas (superficies o hipersuperficies) que consisiten en un conjuntos de valores de parámetros que mantienen constante un atributo. Si y bajo que circunstancias los conjuntos de nivel de los atributos de las neuronas individuales son conservados en las redes de las que forman parte y qué conjuntos de nivel de los atributos a nivel de red emergen no se entiende bien.

En este trabajo describimos un modelo canónico (C-) para LSs de oscillaciones para una célula individual, que exhibe un amplia rango de patrones oscilatorios realistas. El modelo puede ser considerado como una idealización de los familiares modelos de dos dimensiones basados en conductancias. Bajo ciertas condiciones, los LSs de una célula individual son preservados en redes compuestas por C-células. Además, nuevos LSs emergen en estas redes. Los caracterizamos in términos de los LSs de la célula individual y los parámetos de conectividad tanto en redes homogeneas como en heterogeneas donde los células son identas o no, respecticvamente, dentro del LS considerado.
\vspace*{20mm}
\chapter{Conclusion}
\label{sec:con}
\section{Discussion}
In Chapter \ref{sec:results1}, we showed that attribute LSs are preserved in type-I and type-II heterogeneous networks only when individual cells belong to the same frequency LS. However, only when the network is type-I heterogeneous (cells belong to the same frequency and amplitude LSs) gap-junctions preserve individual attribute LSs. In other words, in networks in which cells belong to different amplitude LSs (type-II heterogeneous) gap-junctions do not preserve individual attribute LSs. In this case, the self-connectivity of each cell has to be readjusted in order to guarantee individual LSs preservation. More specifically, the cell with lower individual amplitude value needs higher self-inhibition, whereas the cell with higher individual amplitude value needs lower self-inhibition.

Moreover, we showed that as the network is more complex (it evolves from the type-I heterogeneous case to the most general type-II heterogeneous case), the symmetrical connectivity structure preserving LSs (gap-junctions) is broken until the point in which the network connectivity is unable to continue preserving individual attribute LSs (type-II with cells belonging to different frequency LSs). One question arises: taking into account that the single neuron model considered has a high degree of symmetries, which most probably lead to the fact that the symmetrical gap-junction connectivity preserve attribute LSs in the type-I heterogeneous networks, what would happen in neuronal models with lower degree of symmetries?. 

It might be reasonably that when symmetries are broken, the LS structure evolve from the “organized” structure shown in the $\Lambda \Omega_{2}$ model, to a “non-organized” structure in which each model parameter affects the value of any attribute. It would be something similar to what happens when the individual cell is self-connected (Chapter \ref{sec:results2}). In this scenario, it is likely that the whole individual LS structure will not be preserved, but some attribute LSs could be preserved, as it was shown in \cite{Iii2019}.

Results from \cite{Iii2019} shown that frequency LSs (on a two-dimensional parameter space) in two electrically (gap-junctions) coupled realistic biophysical networks (with cells belonging to the same frequency LS) are preserved. We note that the same feature is observed in the two-cell network, since attribute LSs are preserved in type-I heterogeneous networks.

In Chapter \ref{sec:results2}, we computed several network attribute LSs on different parameter spaces, involving the connectivity parameter space or the intrinsic parameter space of a single cell. Moreover, it was mentioned that more total-degenerated network LSs could be found in other parameter spaces, for example parameter spaces involving both connectivity parameters and intrinsic parameters. For the sake of simplicity, we reduced our analysis to study until 4-dimensional parameter spaces, although higher dimensional parameter spaces could be considered. 

Nevertheless, we have verified one of the main prediction done in  \cite{Oly}. They predict that if a particular homeostatic mechanism maintain m independent characteristics (or attributes) of neural activity, then at least m parameters must be changed as a response to a perturbation in one parameter of the system. For instance, 2-dimensional total-degenerated LSs on the intrinsic parameter space in the self-connected cell verify that statement. 

However, two-cell networks seems not to verify predictions in  \cite{Oly}. Several total-degenerated LSs have been computed preserving the amplitude value of each cell and the frequency value of each cell in the network (an overall of 4 attributes or characteristics). We must mention that predictions in  \cite{Oly} were done under certain hypothesis which might not be satisfied in the two-cell network. 

As per our observations in the two-cell network, if the network shows sustained oscillations, both cells in the network show the same frequency value. This assertion is truly true when the inhibitory/excitatory character of self-connectivity parameter in the network is the same. When this is not the case, the network shows “non-standard” oscillations, in which the standard amplitude or frequency considered in this work might not be well-defined. By an “standard” oscillation we refer then to a sinusoidal-like oscillation. 

Since the frequency value of each cell in the network is the same, a network frequency is well- defined. If the network frequency is considered as a unique attribute regarding frequencies, then predictions in  \cite{Oly} are verified in most LSs computed in Chapter \ref{sec:results2}. Then, the only case in which prediction in  \cite{Oly} are not verified is in symmetrical homogeneous networks. Here, 2-dimensional total-degenerated attribute LSs were found on the connectivity parameter space. These total-degenerated LSs preserve the amplitude value of each cell and the network frequency. Similarly, predictions are verified if it is considered the network amplitude (well-defined) as a unique attribute. Interestingly, although in symmetrical type-I or type-II heterogeneous networks there is also a well-defined network amplitude, the amplitude of each cell should be considered as independent attributes in those cases.

All considered, the type of network (homogeneous or heterogeneous) and the connectivity network architecture seem to affect predictions in  \cite{Oly}.

Finally, we discuss to what extent intrinsic parameters of a cell in the two-cell network can be changed maintaining network attributes constant. It was shown that the two-cell network shows 1-dimensional total-degenerated LSs on the intrinsic parameter space of a given cell in the network. We note that in the self-connected cell, the same total-degenerated LSs were 2-dimensional. Therefore a reduction of one dimension is observed when the network incorporates an addition cell. One question arises: what would happen in a three-cell network?. Most likely, since another attribute (the amplitude of the additional cell) should be preserved in the total-degenerated LS, intrinsic parameters of only a given cell could not be changed maintaining network attributes constant. In other words, it is likely that total-degenerated LSs on the intrinsic parameter space of a single cell could not be found and intrinsic parameters of more than one cell in the network should be changed in order to preserve network attributes (the network frequency and the amplitude of each cell). The same prediction could be done for more complex networks.

Therefore, if a given target activity level was characterized by a total-degenerated network LS of the type considered in this work, the corresponding activity-dependent homeostatic regulation mechanism at the single neuron level would be closely related with a more general mechanism at the network level involving other cells. However more questions arises: what would be the role of synaptic parameters (connectivity parameter) in such a homeostatic level at the network level or what do we exactly mean by a network LS?

\section{Future Work}
Some open questions were given in the previous section. They constitute future lines of research. Furthermore, there are other challenging and open problems.

Most interestingly, it is the connection between homeostatic rules and attribute LSs and how homeostatic rules could be encoded at the model level. In this respect, the connection between data analysis and modelling plays a fundamental role. How could one develop methods for the disambiguation of degeneracy. In Chapter \ref{sec:methods}, it has been suggest a way of decode degeneracy in $\Lambda \Omega_{2}$ systems. Future work could be focus on the development of new techniques which might involve both data analysis and modelling.

It could also be interesting, to study more in detail the connection between the model symmetries and the preservation of LSs. In this regard, a set of models with different degree of symmetries could be considered. Network involving more cells could be considered, although the basic mechanism is expected to be found on single networks.

Finally, new research could also be focused on developing robust, non model-dependent and optimized algorithms to compute LSs on a given parameter space in any parameter model (or at least family of models).  

\section{Personal Conclusion}
During this research experience I have learned about computational neuroscience. I had the opportunity to attend a course about computational neuroscience, in which I knew about the mathematical tools used in this field. I also had the opportunity to attend several meetings and conferences, which made me know about some state of-the-art research and open problems in neuroscience.

Moreover, I have realized the strong connection between modelling and experiments/data. I truly find this field of study quite interesting.

In parallel with the development of this project, I also worked on another problem whose goal was to understand the pathological rhythms found in the basal ganglia. I learned about complex neuronal networks and also realistic neuronal patterns and data.

Furthermore, I have done some presentations (Dana Knox Student Research Showcase, CNS annual meeting,...) which, with no doubts, have contributed to the improvement of my personal skills.

All considered, and taking into that the pandemic made impossible to attend physically the New Jersey Institute of Technology (NJIT), it has been quite an enriching experience. 

Thus, I would like to truly thank Horacio G. Rotstein for making all this possible.